\documentclass[UTF8]{ctexart}
\usepackage{color}
\usepackage{algorithm}  
\usepackage{algpseudocode}  
\usepackage{amsthm,amsmath,amssymb}
\usepackage{mathrsfs}
\usepackage[colorlinks,linkcolor=red]{hyperref}

\usepackage{ctex}
\usepackage{graphicx}
\usepackage{caption2}
\usepackage{subfigure}
\usepackage{float}

\usepackage{tablists}
\restorelistitem

\renewcommand{\algorithmicrequire}{\textbf{Input:}}
\renewcommand{\algorithmicensure}{\textbf{Output:}}

\usepackage{tikz}
\usepackage{verbatim}
\usetikzlibrary{trees}

\title{多种显隐式时间积分器求解弹簧质点系统}
\author{余畅 \\ 2019091621002, 信息与软件工程学院}
\begin{document}
\maketitle

\section{摘要}
\textbf{关键词:弹簧质点系统,欧拉法,Runge-Kutta,Jacobi迭代,Gauss-Seidel迭代} \par
弹簧质点系统作为简单的物理模型,在许多仿真模拟中都有着广泛的应用。本文着重推导了弹簧质点系统的数值模拟方法,分析显式隐式欧拉法的数值稳定性,并在最终实现了多种显隐式时间积分器求解弹簧质点系统。最终通过对各个仿真结果和性能的比较,得出了较为优越的方法。

\section{背景介绍}
弹簧质点模型是利用牛顿运动定律来模拟物体变形的方法。其核心是把质点之间用无质量的、自然长度不为零的弹簧连接后,用微分方程来的形式表示。该模型结合一些物体现实生活中的物理属性, 如质量、硬度和弹性等材料属性, 运用力学理论建立起微分方程。弹簧质点模型在实际仿真中有许多应用,如结合有限元方法对各种类型的布料或者毛发进行仿真,游戏和影视中常常使用其来解算布料和网格模型的形变,无论是离线模拟还是在线模拟都能取得很好的视觉效果。\par
本文从弹簧质点模型实际的力学模型出发,使用已学习的微积分和线性代数知识对各种显式隐式时间积分器进行推导,作为下一步仿真模拟的数学依据。最后使用嵌入在 python 中的高性能并行计算语言 Taichi,实现最后的视觉仿真效果。

\section{模型推导}

\subsection{受力分析}

弹簧质点的运动遵循牛顿第二定律。对模型整体进行受力分析,其同时受到内力和外力的影响,其中内力包括弹簧的弹性力和阻尼力,外力则是重力、空气阻力、与边界碰撞产生的弹力或是人为对系统的扰动等;对每一个质点进行受力分析,其受到的外力包括与其直接相连的弹簧提供的拉力和其他力。由此可见,弹簧质点系统将动能和重力势能储存在质点中,将弹性势能储存在弹簧中。\par

\subsubsection{弹性力}

弹簧的弹性力遵从 Hooke 定律,即在弹性极限内,弹性物体的应力与应变成正比。本文的建模中假设弹簧为理想情况,即弹簧可无限拉伸或者收缩。若质点 $i$ 和质点 $j$ 之间存在弹簧相连,则弹性力可以表示为:

\begin{large}
\begin{equation}
\begin{split}
&\mathbf{f}_i = \mathbf{f}^s(\mathbf{x}_i, \mathbf{x}_j)=k_s \widehat{\mathbf{x}_{ij}} (\lvert \mathbf{x}_{ij} \rvert - l_0) \\
&\mathbf{f}_j = \mathbf{f}^s(\mathbf{x}_j, \mathbf{x}_i)=-\mathbf{f}^s(\mathbf{x}_i, \mathbf{x}_j)=-\mathbf{f}_i
\end{split}
\end{equation}
\end{large}

其中 \begin{large} $\lvert \mathbf{x}_{ij} \rvert = \lvert \mathbf{x}_{j} - \mathbf{x}_{i} \rvert$ \end{large},\begin{large}$\widehat{\mathbf{x}_{ij}} = \cfrac{\mathbf{x}_{j} - \mathbf{x}_{i}}{ \lvert \mathbf{x}_{j} - \mathbf{x}_{i} \rvert}$ \end{large},\begin{large}$ \mathbf{x}_i $\end{large} 表示质点 $i$ 的位置矢量,$l_0$ 表示弹簧的自然长度。 \par

\subsubsection{阻尼力}

考虑到现实中,弹簧在振动过程中会由于摩擦力、空气阻力等损耗振动幅度不断衰减。物理学和工程学上,阻尼的力学模型一般是一个与振动速度大小成正比,与振动速度方向相反的力,即本文采用的线性粘性阻尼模型${}^{[4]}$,可以表示为:

\begin{large}
\begin{equation}
\begin{split}
&\mathbf{f}_i = \mathbf{f}^d(\mathbf{x}_i, \mathbf{v}_i, \mathbf{x}_j, \mathbf{v}_j)=-k_d \widehat{\mathbf{x}_{ij}} (\widehat{\mathbf{x}_{ij}} \cdot \mathbf{v}_{ij}) \\
&\mathbf{f}_j = \mathbf{f}^d(\mathbf{x}_j, \mathbf{v}_j, \mathbf{x}_i, \mathbf{v}_i)=-\mathbf{f}_i
\end{split}
\end{equation}
\end{large}

其中 \begin{large}$\mathbf{v}_{ij} = \mathbf{v}_j - \mathbf{v}_i$\end{large},$k_d$ 表示阻尼系数,$\mathbf{v}_i$ 表示质点 $i$ 的速度矢量。 \par

\subsubsection{其他力}

本文建模时采用理想的外部条件,仅考虑大小方向不随时间变化的常力 \begin{large}$\mathbf{G}_i = m_i g$\end{large} \par

\subsection{显式求解}

解算弹簧质点模型的关键是牛顿运动定律 $\mathbf{f} = m \ddot{\mathbf{x}}$。将该二阶微分方程改写为两个一阶微分方程${}^{[9]}$:

\begin{large}
\begin{equation}
\begin{split}
&\dot{\mathbf{v}} = \mathbf{f}(\mathbf{x}, \mathbf{v}) / m \\
&\dot{\mathbf{x}} = \mathbf{v}
\end{split}
\end{equation}
\end{large}

则现问题为令 $\mathbf{x}$ 和 $\mathbf{v}$ 同时对时间 $t$ 进行数值积分。其中最简单的方法称为显式欧拉法(Explicit Euler's Method)。

\subsubsection{显式欧拉法(Explicit Euler's Method)}

令 $\mathbf{x}$ 的初始值为 $\mathbf{x}_0 = \mathbf{x}(t_0)$,$\mathbf{v}$ 的初始值为 $\mathbf{v}_0 = \mathbf{v}(t_0)$。我们需要估计的是此后时间 $t_0 + h$ 时刻的 $\mathbf{x}(t_0+h)$ 和 $\mathbf{v}(t_0 + h)$,其中 $h$ 是一个积分步长参数。欧拉法简单地通过在导数方向乘上 $h$ (沿着切线斜率方向移动)来计算 $\mathbf{x}(t_0+h)$ 和 $\mathbf{v}(t_0 + h)$,求解上述微分方程可以得到:

\begin{large}
\begin{equation}
\begin{split}
& \mathbf{v}(t_0 + h) = \mathbf{v}(t_0) + h \mathbf{f}(\mathbf{x}(t_0), \mathbf{v}(t_0)) / m \\
& \mathbf{x}(t_0 + h) = \mathbf{x}(t_0) + h \mathbf{v} (t_0)
\end{split}
\end{equation}
\end{large}

下面对欧拉法做局部截断误差(local truncation error)分析${}^{[9]}$,假设 $v(t)$ 光滑,对其泰勒展开得: \par

\begin{large}
\begin{equation}
\mathbf{v}(t_0+h) = \mathbf{v}(t_0) + h \dot{\mathbf{v}} + \frac{h^2}{2!} \ddot{\mathbf{v}}(t_0) + 
\frac{h^3}{3!} \dddot{\mathbf{v}}(t_0)+...+ \frac{h^n}{n!} \frac{\partial^n \mathbf{v}}{\partial t^n} + ...
\end{equation}
\end{large}

移项可知,欧拉法的误差项为 $O(h^2)$,相较于其他一些高阶方法,其数值准确度较低。 \par

实际仿真过程中,本文利用 \begin{large} $E = \sum_{i} m_ig\mathbf{x}_y + \cfrac{1}{2} m_i \mathbf{v}_{i}^{2} + \cfrac{1}{2} k_s \mathbf{x}_{i}^{2}$ \end{large} 来拟合弹簧质点系统整体的机械能,发现随着迭代次数的增加,系统整体的能量错误的随着时间 $t$ 的增大而不断增大,远达不到预期的准确度,且在阻尼系数 $k_d$ 较小时,误差会过快的累积使得系统爆炸。

显式欧拉法最大的优势是积分表达式简单且易于快速求解,但是其数值精度的误差较大,且具有较大的数值不稳定性,所以只适合较小的积分步长 $h$,其中保持系统稳定的 $h$ 的上界和弹簧的劲度系数 $k_s$ 有关 ${}^{[1]}$,为 $h \le c \sqrt{\cfrac{m}{k_s}}$,$c$ 为常数且 $c \sim 1$。直观理解来看,弹簧振子的周期为 $\mathbf{T} = 2 \pi \sqrt{\cfrac{m}{k_s}}$,如果不能完整模拟弹簧一个周期或半周期的振荡,其数值积分解就会发散而导致系统爆炸。如果弹簧的劲度过大接近于刚体,$h$ 将限制在一个较小的范围里,相同的计算时长只能模拟的更小的时间跨度,极大地影响了计算效率。 \par

对显式欧拉法进行简单改进,可以得到半隐式欧拉法(Semi-implicit Euler's Method)。

\subsubsection{半隐式欧拉法(Semi-implicit Euler's Method)}

\begin{large}
\begin{equation}
\begin{split}
& \mathbf{v}(t_0 + h) = \mathbf{v}(t_0) + h \mathbf{f}(\mathbf{x}(t_0), \mathbf{v}(t_0)) / m \\
& \mathbf{x}(t_0 + h) = \mathbf{x}(t_0) + h \mathbf{v} (t_0 + h)
\end{split}
\end{equation}
\end{large}

式 $(6)$ 和式 $(5)$ 的不同之处仅在于对 $\mathbf{x}$ 积分时使用的是 $\mathbf{v} (t_0 + h)$ 而不是 $\mathbf{v} (t_0)$,但是在数值准确性上有了本质的提升。尽管和显式欧拉法一样是 $O(h^2)$ 的误差项,但半隐式欧拉法本质上是一种辛算法(symplectic integrator),因而可以几乎保持能量守恒(当哈密顿量与时间无关时)。

\subsubsection{中点法(Midpoint Method)}

注意到显式欧拉法和半隐式欧拉法误差项为 $O(h^2)$ 的原因在于在 $(5)$ 式中,我们仅将对 $\mathbf{v}(t_0+h)$ 的估计局部截断为 $\mathbf{v}(t_0) + h \dot{\mathbf{v}}(t_0)$,如果我们能像计算 $\dot{\mathbf{v}}$ 一样计算 $\ddot{\mathbf{v}}$,则能通过在泰勒展开中额外保留一项达到 $O(h^3)$ 的准确度:

\begin{large}
\begin{equation}
\mathbf{v}(t_0+h) = \mathbf{v}(t_0) + h \dot{\mathbf{v}}(t_0) + \frac{h^2}{2} \ddot{\mathbf{v}}(t_0) + O(h^3)
\end{equation}
\end{large}

注意到 \begin{large}$\dot{\mathbf{v}}(t_0)=\mathbf{f}(\mathbf{x}(t_0), \mathbf{v}(t_0))$\end{large},用链式求导法则得到:

\begin{large}
\begin{equation}
\ddot{\mathbf{v}} = \frac{\partial \mathbf{f}}{\partial \mathbf{v}} \dot{\mathbf{v}} = \mathbf{f_v} \mathbf{f}
\end{equation}
\end{large}

为了避免进行昂贵和复杂的对 $\mathbf{f}$ 进行直接求导,我们可以近似的用二阶项来估计 $\mathbf{f}$,然后将其代入式 $(7)$ 中,留下 $O(h^3)$ 的误差项。对此我们对 $\mathbf{f}$ 进行另一次泰勒展开得到:

\begin{large}
\begin{equation}
f(\mathbf{x_0}, \mathbf{v_0} + \Delta \mathbf{v}) = f(\mathbf{x_0}, \mathbf{v_0}) + \Delta \mathbf{v} f_v(\mathbf{x_0}, \mathbf{v_0}) + O(\Delta \mathbf{v}^2)
\end{equation}
\end{large}

首先引入 \begin{large} $\ddot{\mathbf{v}}$ \end{large} 通过选择:

\begin{large}
\begin{equation}
\Delta \mathbf{v} = \frac{h}{2} f(\mathbf{x}_0)
\end{equation}
\end{large}

代入得:

\begin{large}
\begin{equation}
\begin{split}
f(\mathbf{x_0}, \mathbf{v_0} + \frac{h}{2} f(\mathbf{x}_0)) &= f(\mathbf{x_0}, \mathbf{v_0}) + \frac{h}{2} f(\mathbf{x}_0) f_v(\mathbf{x_0}, \mathbf{v_0}) + O(h^2) \\ &= f(\mathbf{x_0}, \mathbf{v_0}) + \frac{h}{2} \ddot{\mathbf{v}}(t_0) + O(h^2)
\end{split}
\end{equation}
\end{large}

其中 \begin{large} $\mathbf{x_0} = \mathbf{x}(t_0)$ \end{large},\begin{large} $\mathbf{v_0} = \mathbf{v}(t_0)$ \end{large}。在等式两侧同时乘上 $h$ (其中 $O(h^2)$ 的误差项变为 $O(h^3)$),整理可得:

\begin{large}
\begin{equation}
\frac{h^2}{2} \ddot{\mathbf{v}} + O(h^3) = h(f(\mathbf{x}_0, \mathbf{v}_0 + \frac{h}{2} f(\mathbf{x}_0, \mathbf{v}_0)) - f(\mathbf{x}_0, \mathbf{v}_0))
\end{equation}
\end{large}

将 $(12)$ 式等式右侧代入式 $(7)$ 可得到更新过的公式:

\begin{large}
\begin{equation}
\mathbf{v}(t_0 + h) = \mathbf{v}(t_0) + hf(\mathbf{x}_0, \mathbf{v}_0 + \frac{h}{2} f(\mathbf{x}_0, \mathbf{v}_0))
\end{equation}
\end{large}

直观理解来看,中点法先执行一次步长为\begin{large} $\frac{h}{2}$\end{large}的标准欧拉法,然后取所移动到位置的导数来从原位置重新执行一次移动,由于取导数的位置在标准欧拉法位置的中点,故被称为中点法,其误差项为 $O(h^3)$,但是一次迭代需要计算两次 $\mathbf{f}$。在实际仿真的过程中,中点法的计算时间开销大约是显式/半隐式欧拉法的两倍,但是在处理弹簧高频振荡时相比后者有更高的准确性,且相比后者在弹簧劲度大时更容易保持系统的稳定性。

\subsubsection{龙格-库塔法(Runge-Kutta Method)}

中点法通过在泰勒展开中多保留一项,将误差项从 $O(h^2)$ 降到了 $O(h^3)$,一般地,如果能在泰勒展开中保留更多项,则可以得到更高阶的误差项,这就是龙格-库塔法(Runge-Kutta Method)的思想${}^{[9]}$,特殊地,中点法是二阶 Runge-Kutta 积分。工业和仿真中一般广泛采用的是四阶 Runge-Kutta 积分,其误差项为 $O(h^5)$:

\begin{large}
\begin{equation}
\begin{matrix}
\mathbf{v}_1 = \mathbf{v}^t & \mathbf{a}_1 = f(\mathbf{x}^t, \mathbf{v}^t) / m \\
\mathbf{v}_2 = \mathbf{v}^t + \cfrac{h}{2} \mathbf{a}_1 & \mathbf{a}_2 = f(\mathbf{x}^t + \cfrac{h}{2} \mathbf{v}_1, \mathbf{v}^t + \cfrac{h}{2} \mathbf{a}_1) / m \\
\mathbf{v}_3 = \mathbf{v}^t + \cfrac{h}{2} \mathbf{a}_2 & \mathbf{a}_3 = f(\mathbf{x}^t + \cfrac{h}{2} \mathbf{v}_2, \mathbf{v}^t + \cfrac{h}{2} \mathbf{a}_2) / m \\
\mathbf{v}_4 = \mathbf{v}^t + h \mathbf{a}_3 & \mathbf{a}_4 = f(\mathbf{x}^t + h \mathbf{v}_3, \mathbf{v}^t + h \mathbf{a}_3) / m \\
\mathbf{x}^{t+1} = \mathbf{x}^{t} + \cfrac{h}{6}(\mathbf{v}_1 + 2\mathbf{v}_2 + 2\mathbf{v}_3 + \mathbf{v}_4)
& \mathbf{v}^{t+1} = \mathbf{v}^{t} + \cfrac{h}{6}(\mathbf{a}_1 + 2\mathbf{a}_2 + 2\mathbf{a}_3 + \mathbf{a}_4)
\end{matrix}
\end{equation}
\end{large}

其中 $\mathbf{x}^t = \mathbf{x}(t_0)$,$\mathbf{v}^t = \mathbf{v}(t_0)$,$\mathbf{x}^{t+1} = \mathbf{x}(t_0 + h)$,$\mathbf{v}^{t+1} = \mathbf{v}(t_0 + h)$。

龙格-库塔法最大的优势在于选择不同的展开阶数,其误差项也可以达到任意高的阶数,但由于运算复杂度的限制,其并不能展开到很高的程度,本文使用的就是最常用 $4$ 阶展开。但本文使用的龙格-库塔法仍然局限在显式积分器的范畴内,故仍然无法避免使用小的时间步长和在弹簧劲度系数 $k_d$ 过大的情况下产生数值爆炸。

\subsection{隐式求解}

隐式欧拉积分比显式欧拉积分具有更高的稳定性,弹簧质点模型的隐式欧拉积分表达式为:

\begin{large}
\begin{equation}
\begin{pmatrix}
\mathbf{v}^{t+h} \\
\mathbf{x}^{t+h}
\end{pmatrix}
=
\begin{pmatrix}
\mathbf{v}^{t} + h \mathbf{M}^{-1} \mathbf{f}^{t+h} \\
\mathbf{x}^{t} + h \mathbf{v} ^ {t+h}
\end{pmatrix}
\end{equation}
\end{large}

将 $ \mathbf{f}^{t+h}$ 一阶泰勒展开 \begin{large} $ \mathbf{f}^{t+h} =  \mathbf{f}^{t} + \cfrac{\partial \mathbf{f}}{\partial \mathbf{x}} \Delta \mathbf{x} + \cfrac{\partial \mathbf{f}}{\partial \mathbf{v}} \Delta \mathbf{v}$\end{large},代入式 $(15)$ 得到${}^{[2]}$:

\begin{large}
\begin{equation}
\Delta \mathbf{v} = h \mathbf{M}^{-1} \left ( \mathbf{f}^{t} + \cfrac{\partial \mathbf{f}}{\partial \mathbf{x}} \Delta \mathbf{x} + \cfrac{\partial \mathbf{f}}{\partial \mathbf{v}} \Delta \mathbf{v}  \right ) = 
h \mathbf{M}^{-1} \left ( \mathbf{f}^{t} + \cfrac{\partial \mathbf{f}}{\partial \mathbf{x}} (\mathbf{v}_t + \Delta \mathbf{v}) + \cfrac{\partial \mathbf{f}}{\partial \mathbf{v}} \Delta \mathbf{v}  \right )
\end{equation}
\end{large}

整理得:

\begin{large}
\begin{equation}
\left ( \mathbf{I} - h \mathbf{M}^{-1} \cfrac{\partial \mathbf{f}}{\partial \mathbf{v}} - h^2 \mathbf{M}^{-1} \cfrac{\partial \mathbf{f}}{\partial \mathbf{x}} \right ) \Delta \mathbf{v} =
h  \mathbf{M}^{-1} \left ( \mathbf{f}^{t} + h \cfrac{\partial \mathbf{f}}{\partial \mathbf{x}} \mathbf{v}^t \right )
\end{equation}
\end{large}

将 $(17)$ 式子两边同时乘以 $\mathbf{M}$:

\begin{large}
\begin{equation}
\left ( \mathbf{M} - h \cfrac{\partial \mathbf{f}}{\partial \mathbf{v}} - h^2 \cfrac{\partial \mathbf{f}}{\partial \mathbf{x}} \right ) \Delta \mathbf{v} =
h \left ( \mathbf{f}^{t} + h \cfrac{\partial \mathbf{f}}{\partial \mathbf{x}} \mathbf{v}^t \right )
\end{equation}
\end{large}

其中 $\mathbf{M}$ 是基于仿真维度以 $3 \times 3$ 或 $2 \times 2$ 对角矩阵为子矩阵的对角矩阵。\begin{large}$\cfrac{\partial \mathbf{f}}{\partial \mathbf{x}}$ \end{large}和 \begin{large} $\cfrac{\partial \mathbf{f}}{\partial \mathbf{v}}$ \end{large}为子矩阵为 $3 \times 3$ 或 $2 \times 2$ 的对称稀疏 Jacobian 矩阵。若质点 $i$ 和质点 $j$ 之间有弹簧相连,则相应的子矩阵元不为 $0$。 \par

已知向量模对向量的导数为:\begin{large}$ \cfrac{\partial \lvert \mathbf{x} \rvert}{\partial \mathbf{x}} = \left ( \cfrac{\mathbf{x}}{\lvert \mathbf{x} \rvert} \right )^T = {\hat{\mathbf{x}}}^T $\end{large},其中 
$\hat{\mathbf{x}} = \cfrac{\mathbf{x}}{\lvert \mathbf{x} \rvert}$,单位向量对向量的导数为:\begin{large}$ \cfrac{\partial \hat{\mathbf{x}}}{\partial \mathbf{x}} = \cfrac{\mathbf{I} \lvert \mathbf{x} \rvert - \mathbf{x} \cdot {\hat{\mathbf{x}}}^T}{{\lvert \mathbf{x} \rvert}^2} = \cfrac{\mathbf{I} - \hat{\mathbf{x}} \cdot {\hat{\mathbf{x}}}^T}{\lvert \mathbf{x} \rvert}$ \end{large}。利用以上公式即可推导出 Jacobian 矩阵中子矩阵的表达式:\par

\begin{large}
\begin{equation}
\begin{split}
& \cfrac{\partial \mathbf{f}_i^s}{\partial \mathbf{x}_i} = k_s \left [ \cfrac{\widehat{\mathbf{x}_{ij}} \cdot \widehat{\mathbf{x}_{ij}}^T - \mathbf{I}}{\lvert \mathbf{x}_{ij} \rvert} (\lvert \mathbf{x}_{ij} \rvert - l_0) -  \widehat{\mathbf{x}_{ij}} \cdot \widehat{\mathbf{x}_{ij}}^T \right ] = 
- \cfrac{\partial \mathbf{f}_j^s}{\partial \mathbf{x}_i} = 
- \cfrac{\partial \mathbf{f}_i^s}{\partial \mathbf{x}_j} = 
\cfrac{\partial \mathbf{f}_j^s}{\partial \mathbf{x}_j} \\
& \cfrac{\partial \mathbf{f}_i^s}{\partial \mathbf{v}_i} = \mathbf{0}
\end{split}
\end{equation}
\end{large}

\begin{large}
\begin{equation}
\begin{split}
& \cfrac{\partial \mathbf{f}_i^d}{\partial \mathbf{x}_i} = -k_d \left[ 
(\widehat{\mathbf{x}_{ij}}^T \cdot \widehat{\mathbf{v}_{ij}} \cdot \mathbf{I} +
\widehat{\mathbf{x}_{ij}} \cdot \widehat{\mathbf{v}_{ij}}^T
) \cdot
\cfrac{\widehat{\mathbf{x}_{ij}} \cdot \widehat{\mathbf{x}_{ij}}^T - \mathbf{I}}{\lvert \mathbf{x}_{ij} \rvert}
\right] \\
& \cfrac{\partial \mathbf{f}_i^d}{\partial \mathbf{v}_i} = k_d \widehat{\mathbf{x}_{ij}} \cdot \widehat{\mathbf{x}_{ij}}^T
\end{split}
\end{equation}
\end{large}

令 \begin{large}$ \mathbf{A} = \left ( \mathbf{M} - h \cfrac{\partial \mathbf{f}}{\partial \mathbf{v}} - h^2 \cfrac{\partial \mathbf{f}}{\partial \mathbf{x}} \right ) $\end{large}, \begin{large} $\mathbf{b} = h \left ( \mathbf{f}^{t} + h \cfrac{\partial \mathbf{f}}{\partial \mathbf{x}} \mathbf{v}^t \right )$ \end{large},则问题转化为求线性方程组 \begin{large} $\mathbf{Ax = b}$ \end{large}。一个直观的想法是对 $\mathbf{A}$ 取逆,答案就是 $\mathbf{A^-1b}$,然而,在仿真过程中,矩阵求逆是一个昂贵的操作${}^{[4]}$,时间复杂度可以达到 $O(n^3)$(非并行条件下),其中 $n$ 为矩阵的阶数。此外,原问题中,由于并不是每对质点之间都有弹簧相连,\begin{large} $ \mathbf{M} $ \end{large},\begin{large} $\cfrac{\partial \mathbf{f}}{\partial \mathbf{v}}$ \end{large} 和 \begin{large} $\cfrac{\partial \mathbf{f}}{\partial \mathbf{v}}$ \end{large} 都是稀疏矩阵,其线性运算后的结果 $\mathbf{A}$也是稀疏矩阵,故而所需的存储空间仅和矩阵中非零元的个数相关,远小于 $O(n^2)$,可以用稀疏数据结构存储,反之矩阵求逆之后,稀疏性被破坏,非零元的个数达到 $O(n^2)$,无法直接储存。本文通过实际代码仿真测试,在质点数量达到 $10^2$ 级别时,就无法进行实时的仿真。总结来说,直接求逆操作对计算时间和计算空间都提出了极高的要求,在仿真中较为不实用。 \par

然而,解线性方程组 \begin{large} $\mathbf{Ax = b}$ \end{large} 是一个已知且存在普遍方法的问题,本文采用了以下几种方法求解。

\subsubsection {雅可比迭代法(Jacobi Iterative Method)}

雅可比迭代法首先将矩阵 $\mathbf{A}$ 拆分为 $\mathbf{A=D+E}$,其中 $\mathbf{D}$ 和 $\mathbf{A}$ 对角元相同而其他元为零子矩阵;而 $\mathbf{E=A-D}$,则:

\begin{large}
\begin{equation}
\begin{split}
A&x = b \\
D&x = -Ex + b \\
&x = -D^{-1} E x + D^{-1}b \\
&x = Bx + z
\end{split}
\end{equation}
\end{large}

其中 $B = -D^{-1} E$,$z = D^{-1}b$。由于 $D$ 为对角矩阵,所以其逆很容易求出,由此可以得出迭代求解的方法:

\begin{large}
\begin{equation}
x_{(i+1)} = Bx_{(i)} + z
\end{equation}
\end{large}

令 $x_{(i)} = x + e_{(i)}$,其中 $e_{(i)}$ 为第 $i$ 次迭代后的误差项,则代入式 $(22)$ 可得:

\begin{large}
\begin{equation}
\begin{split}
x_{(i+1)} &= Bx_{(i)} + z \\
&= B(x_{(i)} = x + e_{(i)}) + z \\
&= Bx + z + Be_{(i)} \\
&= x + Be_{(i)} \text{(公式 21)} \\
e_{(i+1)} &= Be_{(i)}
\end{split}
\end{equation}
\end{large}

由式 $(23)$ 可知,每次迭代后 $x$ 的部分不会改变(因为其是迭代式的不动点),但是误差项会每次累乘 $B$,如果 $B$ 的谱半径 $\rho (B) < 1$,那么 $i$ 趋向无穷大时误差项就会趋向于 $0$,此时初始向量 $x_{(0)}$ 的选择并不影响最终的收敛结果。${}^{[8]}$由于仿真过程在时间维度上连续且每次的步长较小,所以本文选择将上次迭代的最终结果作为下一次迭代的初始值。更一般的,若 $B$ 的谱半径 $\rho (B) = \lvert \lambda_{j} \rvert > 1$,且 $v_j$ 为对应的特征向量,则将误差项视为特征向量的线性组合的话,$v_j$ 方向将在迭代中收敛的最慢。

\subsubsection {高斯-赛德尔迭代法(Gauss-Seidel Iterative Method)}

同雅可比迭代法一样,高斯-赛德尔迭代法也是基于矩阵分解的原理,其将矩阵拆分为 $\mathbf{A = D - L - U}$,其中 $\mathbf{D}$ 为对角矩阵,$\mathbf{-L}$ 为对角线以上的上三角矩阵,$\mathbf{-U}$ 为对角线以下的下三角矩阵,则:

\begin{large}
\begin{equation}
\begin{split}
A&x = b \\
(D - L)&x = Ux + b \\
&x =  (D - L)^{-1} Ux + (D - L)^{-1} b \\ 
&x = Bx + z
\end{split}
\end{equation}
\end{large}

其中 $B = (D - L)^{-1} U$,$z = (D - L)^{-1} b$。$(D - L)$ 由于是上三角矩阵,其逆可以在 $O(n^2)$ 的时间复杂度内求得。剩余部分的推导和雅可比迭代法类似。

\subsubsection{小结}

隐式欧拉积分法由于 $\mathbf{f}$ 是基于 $\mathbf{x}(t_0 + h)$ 和 $\mathbf{v}(t_0 + h)$ 来求得的,故又被称为后向欧拉法(Backward Euler's Method)。直观理解来看,就是寻找一个位置,其向负导数梯度的方向回退 $h$ 的步长可以返回到 $\mathbf{x}(t_0)$ 的位置。隐式法在牺牲了一定计算时间的情况下,相比显式法有着额外的数值健壮性,时间步长可以取到更大的范围。在本文的实际仿真测试中,显式积分法在 $h = 10^{-3}$ 级别下可以做到稳定模拟,而隐式积分法在 $5$ 至 $8$ 倍的范围内依然可以做到相对近似拟合,且不受弹簧劲度系数的约束,可以做到稳定模拟刚体。 \par

隐式法虽然对于时间步长的限制更低,但是每一步迭代的代价也更为昂贵,且隐式法的算法相比显式法复杂,计算机和编译器难以进行底层优化,基于速度层面的考虑需要同时考虑两者。此外,本文在仿真过程中发现,隐式法自带数值阻尼(Numrical damping),尽管整个系统会快速收敛到稳定,但是一些显式法可以模拟出的高频振荡现象都会丢失,计算系统总能量后也发现,隐式法总机械能比显式法更低,很难保持能量守恒。在一些游戏模拟和影视模拟中,人们倾向于使用稳定的算法,一般会选用隐式法来求解。但是隐式法也可以继续进行改进,注意到式 $(16)$ 仅对函数 $\mathbf{f}$ 进行一阶泰勒展开,借用龙格-库塔法的思想进行多阶泰勒展开(三阶以上),就可以得到能量守恒的解,但相应的计算开销也非常昂贵,一般仅用于离线工程物理仿真。


\section{代码实现}
本文采用 python / Taichi 语言实现所有代码,对于每种不同的时间积分器,其代码实现框架大体如\textbf{算法 1}所示:

\begin{algorithm}
\caption{Mass-spring system framework}  
\label{alg: Mass-spring system framework} 
\begin{algorithmic} [1] 
    \Function {Mass-Spring-System-Framework}{}
		\For{$frame = 0 \to max\_frames$}
		\State \Call{Process-Input}{$frame$}
		\If{\textbf{not} $paused$}
			\For{$i = 0 \to max\_steps$}
				\State \Call{Substep}{$max\_iteration$}
			\EndFor
		\EndIf
		\State \Call{Process-Output}{$frame$}
		\EndFor
	\EndFunction  
\end{algorithmic}
\end{algorithm}

其中 $max\_frames$ 为程序最长模拟的帧数,$frame$ 为当前模拟的帧数,$max\_steps$ 为每帧需要迭代的次数,$max\_iteration$ 适用于隐式法,为解线性方程组时需要迭代的次数。Process-Input 用于生成处理用户输入和添加/删除质点和弹簧,Process-Output 用于在 GUI 上实时绘制结果,其具体实现与本文关系不大,故不在此赘述。 \par

隐式法有额外构造线性方程组 $\mathbf{Ax=b}$ 的算法,其框架也大体相同,如\textbf{算法2}所示:

\begin{algorithm}
\caption{Substep for Implicit-Method}  
\label{alg: Substep for Implicit-Method} 
\begin{algorithmic} [1] 
    \Function {Substep}{$iteration$}
		\State $\mathbf{M} \gets $ \Call{UpdateMassMatrix}{}
		\State $\cfrac{\partial \mathbf{f}}{\partial \mathbf{x}}$, $\cfrac{\partial \mathbf{f}}{\partial \mathbf{v}} \gets $  \Call{UpdateJacobiMatrix}{}
		\State $\mathbf{A} \gets $ \Call{UpdateAMatrix} {$\mathbf{M}$, $\cfrac{\partial \mathbf{f}}{\partial \mathbf{x}}$, $\cfrac{\partial \mathbf{f}}{\partial \mathbf{v}}$}
		\State $\mathbf{F} \gets $ \Call{UpdateFVector}{}
		\State $\mathbf{b} \gets $ \Call{UpdateBVector}{$\mathbf{F}$}
		
		\State $x \gets x_{0}$
		\For{$step = 0 \to iteration$}
			\State $x \gets $ \Call{Solver}{$x$, $\mathbf{A}$, $\mathbf{b}$}
		\EndFor

	\State \Call{UpdateStep}{$x$}
	\EndFunction  

\end{algorithmic}
\end{algorithm}

其中 Solver 函数为 Jacobi 迭代或者 Gauss-Seidel 迭代。 \par

额外的,为了在仿真过程中更好的观测到弹簧的振荡效应和分析受力,本文使用以下公式给弹簧着色:

\begin{large}
\begin{equation}
\begin{split}
\max(0, &\mathbf{sigmoid(\lvert \mathbf{x}_i - \mathbf{x}_j \rvert - l_0}) \times \mathbf{RED} + \min(0, \mathbf{sigmoid(\lvert \mathbf{x}_i - \mathbf{x}_j \rvert - l_0}) \times \mathbf{GREEN} \\
&\mathbf{sigmoid}(x) = \cfrac{2}{1 + \exp(-x \times k_s \times c)} - 1
\end{split}
\end{equation}
\end{large}

其中 $c$ 为一个常数,为了得到较好的显示效果,本文中的取值为 $0.1$。

本文全部代码托管在:\url{https://github.com/g1n0st/GAMES201/tree/master/hw1}

\section{结果展示}

\begin{figure}[H]
	\raggedleft
    \includegraphics[width=1.1\textwidth]{result1}
    \caption{中间帧}
    \label{Result.1}
\end{figure}

\begin{figure}[H]
	\raggedleft
    \includegraphics[width=1.1\textwidth]{result2}
    \caption{结束帧}
    \label{Result.2}
\end{figure}

从仿真结果来看,显式欧拉法几乎没有稳定性可言,在与地面碰撞后,从能量计算就可以看出误差非常大。从中间帧来看,除了显式欧拉法以外,各算法仿真结果之间的差异还是很小。而在加入高势能的子系统后,系统中出现了高频振荡,半隐式欧拉法表现出了不稳定性,中点法和 Runge-Kutta 法则很好的模拟了弹簧的振荡(从颜色变化可以看出)。Jacobi 迭代和 Gauss-Seidel 迭代由于是隐式法,自带的数值阻尼使得振荡很快收敛。最后的仿真部分模拟了弹簧劲度系数的增大,半隐式欧拉法很快数值爆炸,中点法和 Runge-Kutta 法虽然没有数值爆炸,但是高亮的颜色指示显示出的能量分布表现出拟合误差较大,系统不稳定,隐式法则具有较低的能量,依然保持着稳定性。

从时间复杂度来看,显式方法的优势较为明显,分布在 $1 \sim 3$ 毫秒每帧的范围内,隐式法则为了稳定性牺牲了部分效率,分布在 $10 \sim 20$ 毫秒每帧的范围内。

\section{总结}

本文针对弹簧质点系统受力和求导分析后,使用显式欧拉法,半隐式欧拉法,中点法,Runge-Kutta 法,隐式欧拉法(Jacobi迭代,Gauss-Seidel 迭代)等时间积分器对弹簧质点系统进行求解,并使用 python/Taichi 语言实现了通用的框架,分别实现了每种积分器并通过仿真进行比较,发现在弹簧劲度系数小,时间步长小等较为稳定的情况下,半隐式欧拉法可以在同样较好拟合度的情况下效率最高,而在弹簧劲度系数大,时间步长大等极端条件下,隐式欧拉法可以得到较好的拟合效果而保持系统的稳定性,但同时会出现能量损失的情况。如果需要更高精度的隐式欧拉法,则需要在推导隐式法时使用泰勒展开展开到更高阶,但是同样会需要更高的时间复杂度,仅在需要极高精度的情况下使用。

\section{参考文献}

\begin{small}
\begin{itemize}
\item{[1]} MIT EECS 6.837 Computer Graphics, Particle Systems
and ODE Solvers II,
Mass-Spring Modeling
\item{[2]} Numerical Recipes, 3rd edition 2007 - ch17.5
\item{[3]} Encyclopedia of Applied and Computational Mathematics
Full entry, version of 12 August 2011
Euler methods, explicit, implicit, symplectic
\item{[4]} BARAFF, D., AND WITKIN, A. 1998. Large steps in cloth simulation. In Proceedings of SIGGRAPH 98, ACM Press / ACM SIGGRAPH, Computer Graphics Proceedings, Annual Conference Series, ACM, 43-54.
\item{[5]} DESBRUN, M., SCHRÖDER, P., AND BARR, A. 1999. Interactive animation of structured deformable objects. In Graphics Interface, 1-8.
\item{[6]} VOLINO, P., AND MAGNENAT-THALMANN, N. 2001. Comparing efficiency of integration methods for cloth animation. In Proceedings of the Conference on Computer Graphics International (CGI-01). 
\item{[7]} Yu, W., KANG, T., AND CHUNG, K. 2000. Drape simulation of woven fabrics by using explicit dynamic analysis. Journal of Textile Institute 91 Part 1, 2, 285-301.
\item{[8]} J. Shewchuk. An Introduction to the Conjugate Gradient Method Without the Agonizing Pain. Technical Report CMU-CS-TR-94-125, Carnegie Mellon University, 1994.
\item{[9]} Physically Based Modeling: Principles and Practice (Online Siggraph '97 Course notes)
\end{itemize}
\end{small}

\end{document}



